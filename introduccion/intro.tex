% Introduccion de la monografia
% Probando

\documentclass[12pt]{report}
\usepackage[spanish]{babel}
\usepackage{graphicx}
\begin{document}
\pagestyle{myheadings}
\markboth{Diseño, desarrollo e implementación de Linux Terminal Server Project (Red de Terminales Ligeras) utilizando Tecnología de 
Software Libre en el laboratorio de Computación del Colegio Guardabarranco en la ciudad de Managua}{}
\section{Introduccion}
En la actualidad los avances tecnol\'ogicos han llevado al mundo a lo que hoy en día es la globalizaci\'on, lo cual exige a la persona que est\'a 
laborando en esta sociedad a desenvolverse en distintos ambientes. 

En el caso particular de Nicaragua y en la educaci\'on p\'ublica el hecho de preparar 
acad\'emicamente a los estudiantes en distintos \'ambitos tiene un nivel bajo de prioridad, puesto que las instituciones de enseñanza p\'ublica, como 
escuelas primarias y secundarias, no cuentan con apoyo y capacidad para poder enseñar nuevas herramientas inform\'aticas que son de gran utilidad en 
el \'ambito laboral actual, ya que a trav\'es de la inform\'atica el estudiante puede prepararse intelectualmente en distintas \'areas de estudio. 

Para ayudar a cubrir la necesidad de mejorar la calidad educativa, existe una herramienta desarrollada en base a software libre que permite a las 
entidades educativas el poder de ayudar a sus alumnos a conocer estos ambientes inform\'aticos sin necesidad de que la misma invierta en equipos 
inform\'aticos muy costosos y en licencias de software, facilitando así el acceso a estas tecnolog\'ias para sus estudiantes, esta herramienta se 
llama Linux Terminal Server Project. 

El Linux Terminal Server Project (Proyecto Servidor de Terminales Linux, LTSP), es un proyecto de software libre que permite agregar soporte para 
clientes livianos (thin- clients,terminales tontas) a trav\'es de servidores Linux. Es una soluci\'on econ\'omica y flexible ideal para colegios, 
oficinas, o quien no cuenta con capacidades de adquirir equipos costosos. 

En el mundo de la tecnolog\'ia, mientras m\'as aplicaciones, opciones, manejo, etc. tenga un dispositivo tecnol\'ogico-digital (en nuestro caso un 
computador), m\'as costoso se vuelve. 

En lugar de comprar equipos con gran capacidad de hardware, que por lo general su costo alcanza f\'acilmente \$800 (en nuestro pa\'is), dinero con 
que no cuentan los centros educacionales, tanto de primaria como de secundaria, algo que es tan \'util para los estudiantes en su formaci\'on tanto 
acad\'emica como laboral y que no existen los recursos necesarios para su aprendizaje, con LTSP, esto no es necesario, ya que este proyecto basta 
con un servidor central que hace que un n\'umero de maquinas con casi nada de hardware trabaje sin problemas de manera eficiente. 

Los gobiernos anteriores no han destinado un presupuesto a educaci\'on que ayude a dotar de salones de inform\'atica a los colegios por falta de 
recursos econ\'omicos debido a los costos de hardware actualizado y licencias de software para que estas puedan funcionar.

LTSP, proporciona un modo \'optimo de utilización de recursos, permite ahorrar dinero en licencias (es gratuito), en equipos 
(hardware) y en su administraci\'on, ya que utiliza terminales ligeros conectados a un servidor central. 
 
Este proyecto, por la gran sencillez de la que trata, y de sus bajos costos de implementaci\'on, es aplicable en muchos lugares, tanto empresariales 
(PYME) como educacionales. 

El proyecto se aplicar\'a en el Colegio Publico Guardabarranco en la ciudad de Managua, usando computadoras de bajo recursos y que se encuentren 
inhabilitadas, para ser habilitadas usando software libre que en adelante del documento se explicar\'a sobre la implementaci\'on y el uso del mismo, 
lo que ayudar\'ia a disminuir costos ya que no existir\'ian pagos de licencias y software y no se tendr\'ia que tener hardware actualizado para que 
en lo que en adelante se denominar\'a sal\'on de inform\'atica pueda funcionar con software actual .

\end{document}
